\documentclass[12pt]{article}

\usepackage{amssymb}
\usepackage{amsmath}
\usepackage{graphics}
\usepackage{graphicx}
\usepackage{enumerate}

\title{\textbf{
UNIVERSIDAD NACIONAL SAN CRISTOBAL DE HUAMANGA \\Práctica 01\\

\begin{center}
\thispagestyle{empty}
{\large {\textbf{\sc Universidad Nacional de San Cristóbal de Huamanga}}}\\
\vspace{1.5mm}
{\large {\sc Facultad de Ingeniería de Minas, Geología y Civil}}\\
\vspace{1.5mm}
{\large {\sc Escuela Profesional ciencias físico matemática}}\\
\vspace{1.5mm}
%{\Large {\sc Área de Investigación: Estructuras y Construcciones}}\\
\end{center}
\begin{center}
\includegraphics[scale=0.8]{graficos/UNSCH.png} 




\vspace{-1cm}
\begin{center}
\vfill
{\normalsize {\sc  RESOLUCIONES DE EJERCICIO NÚMERO 1}}\\[0.3cm]

\end{center}


\vspace*{\stretch{1}}
\begin{center}
{\normalsize{\sc  ALUMNO:}}\\[0.3cm]
{\large QUISPE GALINDO, wilmer rodrigo} \\[0.1cm]
\end{center}

\vspace*{\stretch{1}}
\begin{center}
{\normalsize {\sc  PROFESOR:}}\\[0.1cm]
{\large ROMERO PLASENCIA, Jackson} \\[0.1cm]
\end{center}

\vspace*{\stretch{2}}
\begin{center}
\normalsize \textsc{AYACUCHO - PERÚ}
\end{center}

\begin{center}
{\normalsize {\sc  2019}}
\end{center}




\date{}
\begin{document}
\maketitle

\begin{enumerate}
\item  Demuestre que $\mathcal{F}$ es una $\sigma$ - álgebra de subconjuntos de si, y solo si,satisface las siguientes propiedades:
\begin{enumerate}[a)]

\item $\phi \in \mathcal{F}$\\[0.2cm]
Demostración\\[0.2cm]
Si $\Omega \in \mathcal{F}$ y $ \mathcal{F}$ colección cerrada\\[0.2cm]
$\Omega^c = \phi\in\mathcal{F}$\\[0.2cm]

\item A $\in \mathcal{F}\longrightarrow A^{c} \in\mathcal{F}$\\[0.2cm]
Demostración\\[0.2cm]
$A_1\in \mathcal{F} \longrightarrow {A_1^c} \in \mathcal{F} $\\[0.2cm]
$A_2\in \mathcal{F} \longrightarrow {A_2^c} \in \mathcal{F} $\\[0.2cm]


\item $A_1,A_2,... \in \mathcal{F} \longrightarrow\displaystyle\bigcap_{n=1}^{\infty}{A_n \in N}$\\[0.2cm]
Demostración:\\[0.2cm]
Por definición: $\{A_n\}_{n \in N} \in \mathcal{F} \Rightarrow \{A_n\}^c_{n \in N}\in \mathcal{F}$\\[0.2cm]
Entonces $\displaystyle\bigcup_{n=1}^{\infty}{A_n}\in \mathcal{F}$ y a su vez $\displaystyle\bigcup_{n=1}^{\infty}{A_n}^c\in \mathcal{F}$\\[0.2cm]
Luego $\left(\displaystyle\bigcup_{n=1}^{\infty}{A_n^c}\right)^c = \displaystyle\bigcap_{n=1}^{\infty}{A_n \in N}$

\end{enumerate}

\item Sea $\mathcal{F}$ una $\sigma$ - álgebra; demuestre que $\mathcal{F}^c$ es una $\sigma$ - álgebra definida por: $\mathcal{F}^c$ = \{$A^c$: A $\in$ $\mathcal{F}$ \}\\[0.2cm]
Demostración\\[0.2cm]
Como $\mathcal{F}$ es una $\sigma$ - álgebra entonces cumple las siguientes propiedades:\\[0.2cm]
\begin{enumerate}[1.]
\item $\Omega \in \mathcal{F}  $
\item A $\in \mathcal{F}\longrightarrow A^{c} \in\mathcal{F}$
\item $A_1,A_2,... \in \mathcal{F} \longrightarrow\displaystyle\bigcup_{n=1}^{\infty}{A_n \in \mathcal{F}}$
\end{enumerate}
Supongamos que $\mathcal{F}^c$ es un $\sigma$ - álgebra $\Omega \in \mathcal{F}^c  $ que verifica la primera propiedad.\\[0.2cm]
Luego como $\{A_n\}_{n \in N}$ es una sucesión esta definida:\\[0.2cm]
$\{A_n\}\in \mathcal{F}^c \Rightarrow \{A_n\}^c \in  \mathcal{F}^c $ se verifica por la tercera propiedad\\[0.2cm]
Ahora por la tercera propiedad verificamos que:\\[0.2cm]
$A_1,A_2,... \in \mathcal{F}^c\Rightarrow \displaystyle\bigcup_{n=1}^{\infty}{A_n \in \mathcal{F}^c} = \{ A^c/ A \in \mathcal{F}\} $
 

\item Sea $\{A_n\}_{n\in\mathbb{N}}$ la sucesión de eventos, definida por:
\[A_n = \begin{cases} 
      A  & \mbox{si } n= 1,3,5,...        \\[0.5cm]
      A^c  & \mbox{si } n = 2,4,6,...
 \end{cases} \]
Determine el: $\lim\limits_{x \to \infty}A_n$\\[0.2cm]
Solución\\[0.2cm]
$\lim\limits_{n \to \infty}{inf}A_n =\displaystyle\bigcup_{n=1}^{\infty}\displaystyle\bigcap_{k=n}^{\infty}A_k = \displaystyle\bigcup_{n=1}^{\infty}\left(A \cap A^c\right)= \displaystyle\bigcup_{n=1}^{\infty}\phi = \phi $\\[0.2cm]
$\lim\limits_{n \to \infty}{sup}A_n= \displaystyle\bigcap_{n=1}^{\infty}\displaystyle\bigcup_{k=n}^{\infty}A_k = \displaystyle\bigcap_{n=1}^{\infty}\left(A \cap A^c\right)= \displaystyle\bigcap_{n=1}^{\infty}\Omega = \Omega $\\[0.2cm]
$\Rightarrow \lim\limits_{n \to \infty}{inf}A_n \not = \lim\limits_{n \to \infty}{sup}A_n $\\[0.2cm]
$\therefore \lim\limits_{n \to \infty}A_n \nexists$



\item Sea $\{A_n\}_{n\in\mathbb{N}}$ la sucesión de eventos, definida por:
\[A_n = \begin{cases} 
  A  & \mbox{si } \left[-1/n,0\right]  \\[0.5cm]
  A^c  & \mbox{si } \left[0,1/n\right] \end{cases} \]
Determine el: $\lim\limits_{x \to \infty}A_n$

\item Sean $A_1,A_2,...$ eventos aleatorios, demuestre :
\begin{enumerate}[a)]
\item $P\left(\displaystyle\bigcap_{i=1}^{n}{A_i}\right)\geq 1-\displaystyle\sum_{i=1}^{n}$ $P\left(A_{i}\right)^c$\\[0.2cm]
Demostración\\[0.2cm]
Si: $P\left(\displaystyle\bigcap_{i=1}^{n}{A_i}\right)= P\left(\displaystyle\bigcup_{i=1}^{n}{A_i^c}\right)^c$\\[0.2cm]
$P\left(\displaystyle\bigcap_{i=1}^{n}{A_i}\right)= 1- P\left(\displaystyle\bigcup_{i=1}^{n}{A_i^c}\right)$\\[0.2cm]
Recuerde: $P\left(\displaystyle\bigcup_{i=1}^{n}{A_i}\right) \leqslant\displaystyle\sum_{i=1}^{n}$ $P\left(A_{i}\right)$\\[0.2cm]
$\Rightarrow P\left(\displaystyle\bigcap_{i=1}^{n}{A_i}\right)= 1- P\left(\displaystyle\bigcup_{i=1}^{n}{A_i^c}\right) \geqslant 1-\displaystyle\sum_{i=1}^{n}$ $P\left(A_{i}\right)^c $

\item Si $P\left(A_{i}\right) \geq 1-e$ para i=1,2,...,n entonces $P\left(\displaystyle\bigcap_{i=1}^{n}\right)\geq 1-ne$\\[0.2cm]

Demostración\\[0.2cm]
tenemos: $P\left(A_{i}\right)\geqslant 1-e$\\[0.2cm]
	$\Rightarrow e \geqslant 1-P\left(A_{i}\right)$\\[0.2cm]
	$ \displaystyle\prod_{i=1}^{n}e \geqslant \displaystyle\prod_{i=1}^{n}P\left(A_{i}\right)^c$\\[0.2cm]
$ne \geqslant \displaystyle\prod_{i=1}^{n}\left(1- P{\left(A_{i}\right)}\right)\Rightarrow ne\geqslant 1-P\left(\displaystyle\bigcap_{i=1}^{n}A_i\right)$ \\[0.2cm]
$\therefore P\left(\displaystyle\bigcap_{i=1}^{n}A_i\right)\geqslant 1-ne $
			 
		
\item $P\left(\displaystyle\bigcap_{k=1}^{\infty}{A_k}\right) \geqslant 1-\displaystyle\sum_{i=1}^{\infty}P\left(A_{k}^{c}\right) $
\end{enumerate}

\item Demuestre las desigualdades de Boole\\[0.2cm]
$P\left(\displaystyle\bigcup_{n=1}^{\infty}{A_n}\right) \leqslant \displaystyle\sum_{n=1}^{\infty}P\left(A_n\right) $\\[0.2cm]
Por inducción para familia finita queremos demostrar que:\\[0.2cm]
$P\left(A_1\cup A_2\cup A_3...\cup A_n\right) \leqslant P\left(A_1\right)+ P\left(A_2\right)+...+P\left(A_n\right)$	\\[0.2cm]
Luego: para n=1\\[0.2cm]
$P\left(A_1\right)\leqslant P\left(A_1\right) $ se cumple\\[0.2cm]
para n=h\\[0.2cm]
$P\left(A_1\cup A_2\cup A_3...\cup A_h\right)\leqslant P\left(A_1\right)+ P\left(A_2\right)+...+P\left(A_h\right)$ Hipótesis Inductivo.\\[0.2cm]
para n=h+1\\[0.2cm]
$P\left(A_1\cup A_2\cup A_3...\cup A_{h+1}\right)\leqslant P\left(A_1\right)+ P\left(A_2\right)+...+P\left(A_h+1\right)$\\[0.2cm]
Para ello consideramos que:\\[0.2cm]
$x=A_1\cup A_2\cup A_3...\cup A_n$, luego $P\left(x\right) \leqslant P\left(A_1\right)+ P\left(A_2\right)+...+P\left(A_h\right) $\\[0.2cm]
Entonces:\\[0.2cm]
$P\left(x \cup A_{h+1}\right)= P\left(x\right)+P\left(A_{h+1}\right)-P\left(x \cap A_{h+1}\right)$\\[0.2cm]
Luego\\[0.2cm]
$P\left(x\right)+P\left(A_{h+1}\right)-P\left(x \cap A_{h+1}\right)\leqslant P\left(A_1\right)+...+P\left(A_h\right)+P\left(A_{h+1}\right)-P\left(x \cap A_{h+1}\right)$\\[0.2cm]
Así $P\left(x \cup A_{h+1}\right)\leqslant P\left(A_1\right)+...+P\left(A_h\right)+P\left(A_{h+1}\right) $\\[0.2cm]
$P\left(A_1\cup A_2\cup A_3...\cup A_{h+1}\right)\leqslant P\left(A_1\right)+...+P\left(A_h\right)
$\\[0.2cm]
Por tanto:\\[0.2cm]
$P\left(\displaystyle\bigcup_{n=1}^{\infty}{A_n}\right) \leqslant \displaystyle\sum_{n=1}^{\infty}P\left(A_n\right) $


\item Sea $\{A_n\}n \in {N}$ una sucesión de eventos. Demuestre que:
\begin{enumerate}[a)]
\item $\left(\displaystyle\lim_{n\longrightarrow \infty}{inf}A_n\right)^c =\displaystyle\lim_{n\longrightarrow \infty} {sup}{A_n^c} $\\[0.2cm]
Demostración\\[0.2cm]
$\left(\displaystyle\lim_{n\longrightarrow \infty}{inf}A_n\right)^c=\left(\displaystyle\bigcup_{n=1}^{\infty}\displaystyle\bigcap_{k=n}^{\infty}{A_k}\right)^c = \displaystyle\bigcap_{n=1}^{\infty}\displaystyle\bigcup_{k=n}^{\infty}{A_k^c}=\displaystyle\lim_{n\longrightarrow \infty} {sup}{A_n^c}$

\item $\left(\displaystyle\lim_{n\longrightarrow \infty}{sup}A_n\right)^c =\displaystyle\lim_{n\longrightarrow \infty} {inf}{A_n^c} $\\[0.2cm]
Demostración\\[0.2cm]
$\left(\displaystyle\lim_{n\longrightarrow \infty}{sup}A_n\right)^c =\left(\displaystyle\bigcap_{n=1}^{\infty}\displaystyle\bigcup_{k=n}^{\infty}{A_k}\right)^c =\displaystyle\bigcup_{n=1}^{\infty}\displaystyle\bigcap_{k=n}^{\infty}{A_k^c} = \displaystyle\lim_{n\longrightarrow \infty} {inf}{A_n^c} $

\item $P\left(\displaystyle\lim_{n\longrightarrow \infty}{inf}A_n\right) =1-P\left(\displaystyle\lim_{n\longrightarrow \infty}{sup}A_n^c\right)$\\[0.2cm]
Demostración\\[0.2cm]
 $P\left(\displaystyle\lim_{n\longrightarrow \infty}{inf}A_n\right)=\left([P\displaystyle\bigcup_{n=1}^{\infty}\displaystyle\bigcap_{k=n}^{\infty}{A_n}]^c\right)^c = 1-P\left(\displaystyle\bigcup_{n=1}^{\infty}\displaystyle\bigcap_{k=n}^{\infty}{A_n}\right)^c  $\\[0.2cm]
 = $ 1-P\left(\displaystyle\bigcap_{n=1}^{\infty}\displaystyle\bigcup_{k=n}^{\infty}{A_n^c}\right)= 1-P\left(\displaystyle\lim_{n\longrightarrow \infty}{sup}A_n^c\right) $
 

\end{enumerate}

\item Encuentre las condiciones sobre los eventos $A_1$ y $A_2$ para que la siguiente sucesión sea convergente.
\[A_n = \begin{cases} 
     A_1  & \mbox{si } $n es impar$   \\[0.5cm]
      A_2  & \mbox{si }$ n es par$
 \end{cases} \]
Solución:\\[0.2cm]

Para que la sucesión converja tiene que cumplir lo siguiente.

\[\mathop {\lim }\limits_{n \to \infty } \inf {A_n} = \mathop {\lim }\limits_{n \to \infty } \sup {A_n} = A\]

Por tanto la condición para que converja la sucesión es:

\[{A_1} = [\frac{{ - 1}}{n},0] \forall n \in \mathds{N}
]

\[{A_{_2}} = [0,\frac{1}{n}];\forall n \in \mathds{N}
]


por lo que:


\[\mathop {\lim }\limits_{n \to \infty } \inf {A_1} = \mathop {\lim }\limits_{n \to \infty } \sup {A_2} = {0}\]


\end{enumerate}
\end{document}